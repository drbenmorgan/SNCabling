\documentclass[12pt,a4paper]{article}

\usepackage{a4wide}
\usepackage{verbatim}
\usepackage[T1]{fontenc}
\usepackage{ucs}
\usepackage[utf8x]{inputenc}
\usepackage[dvips]{color}
\usepackage{graphicx}
\usepackage{epic}
\usepackage{eepic}
\usepackage{eepicemu}
\usepackage{array}
\usepackage{moreverb}
\usepackage{fancyvrb}
\usepackage{url}
\usepackage{eurosym}
\usepackage{amsmath}
\usepackage{amssymb}
\usepackage{multicol}

\newcommand{\imgpath}{./images}
\newcommand{\samplespath}{./samples}
\newcommand{\pdfteximgpath}{./pdftex}
\newcommand{\pdftextimgpath}{./pdftex_t}

\title{SuperNEMO Demonstrator\\
  Calorimeter High Voltage System (CaloHV)\\
  Cabling scheme and cable labels\\
  version 0.3\\
[DocDB #4760]}
\author{M.Bongrand, Y.Lemi\`ere, F.Mauger}
\date{November 6th, 2018}

%%%%%%%%%%%%%%%%
\begin{document}

\maketitle

\begin{abstract}
  \noindent This document presents the cable labelling convention used
  for  the SuperNEMO  Demonstrator's Calorimeter  High Voltage  System
  (CaloHV).  We reuse  here some informations available  from a couple
  of documents prepared  by Cedric, Mathieu, Christian  with some adaptations
  and addons.

  \vskip 10pt
  \noindent This document and all associated tools
  are hosted at:
  \vskip 5pt
  \url{https://gitlab.in2p3.fr/SuperNEMO-DBD/SNCabling}.
  
\end{abstract}

\tableofcontents
\vfill

\clearpage
\section{Principle}

The SuperNEMO Demonstrator's Calorimeter  High Voltage System (CaloHV)
uses two CAEN High  Voltage crates which host a total  of 24 boards to
distribute HV to 712 PMTs.

Each HV board manages up to 32 channels and is connected to a specific
set of  PMTs through  a pair  of harnesses.   A first  harness, called
\emph{external  harness},  links the  board  (Radial  connector) to  a
single  connector  (Redel  S  Male)   on  the  patch  panel  (external
side). From the  internal face of the patch panel  (Redel S Female), a
new harness, namely the  \emph{internal harness}, routes individual HV
cables  to  the PMTs.   The  end  of  the \emph{internal  harness}  is
designed in  such a  way cables  can be  routed individually  to their
associated PMTs  by splitting in two  wires (HV and ground).   A given
harness aggregates  a set of  cables for PMTs that  are geographically
close to each other, in order to optimize the length of the cables.

In the present  scheme, it has been decided to  identify individual HV
distribution cables using the pin identifiers on the Radial connectors
they are  associated to on  each HV board.  The  pin number on  the HV
board output connector  thus identifies the cable linked  to it.  This
identifier  is  propagated  to  the  pin number  on  the  patch  panel
connectors  then to  the end  of the  cable linked  to the  PMT.  This
enables to build rather simple and comprehensible cabling tables.

\noindent\par Figure \ref{fig:calohv:principle:1}  shows the principle
of the calorimeter  HV distribution.  A dedicated  labelling system is
used to ease the cabling operations (see next sections).

\begin{figure}[h!]
  \begin{center}
    \scalebox{0.75}{\input{\pdftextimgpath/fig-calohv-1.pdftex_t}}
  \end{center}
  \caption{Principle of  the HV  distribution from  the CAEN  HV power
    supplies to the PMTs.}
  \label{fig:calohv:principle:1}
\end{figure}

\clearpage

\section{Addressing objects}

\subsection{Format of a CaloHV label}

Each label to be used for CaloHV cabling will use the following format:

\begin{center}
  \fbox{\texttt{X:}$id_1$.$id_2$\dots$id_n$}
\end{center}
\noindent  where \texttt{X}  is a  single letter  which describes  the
category  of the  labelled  object,  and the  $id_1$.$id_2$\dots$id_n$
sequence is the unique address of the object within its category.  The
$id_x$ tokens are positive integers (possibly zero).  The \emph{colon}
character is  used to separate  the category letter from  the address.
The  sequence  of  identifiers  in  the  address  uses  the  \emph{dot}
character as a separator.


\subsection{HV crates, boards, channel, harnesses and cables}

Each CAEN HV power supply crate belonging to the CaloHV system is installed
in the rack number 2 on the electronics platform.
A HV crate is identified with an unique ID, namely
a number ranging from \texttt{0} to \texttt{1}.  We propose to label a
given HV crate with the following scheme:
\begin{center}
   \fbox{\texttt{C:$crate$}}
 \end{center}
where \texttt{$crate$} is the number of the crate (positive integer).
\vskip     10pt    \par\noindent     Examples:    \fbox{\texttt{C:0}},
\fbox{\texttt{C:1}}.  \par Conventionally, crate \texttt{0} manages HV
for PMTs on the \emph{Italy} side  and crate \texttt{1} manages HV for
PMTs on the \emph{France} side.

\vskip 10pt A HV crate contains up to 16 HV 32-channel boards but only 12
will be  used.  A  HV board  inherits the  number of  the crate  it is
plugged into  and is addressed through  its slot number.  We give a given HV board
a label with the following scheme:
\begin{center}
   \fbox{\texttt{B:$crate$.$board$}}
 \end{center}
where \texttt{$crate$} is the number of the crate and \texttt{$board$}
is  the  number  of  the  board  (slot)  ranging  from  \texttt{0}  to
\texttt{15}.
\vskip 10pt
\par\noindent Examples: \fbox{\texttt{B:0.0}},  \dots
\fbox{\texttt{B:1.11}}.

\vskip 10pt
\par\noindent Up  to 32 HV channels are addressed within a HV board.
 We propose to label a given HV channel with the following scheme:
\begin{center}
   \fbox{\texttt{H:$crate$.$board$.$channel$}}
 \end{center}
where \texttt{$crate$} is the number of the crate,
  \texttt{$board$} is the number of the crate
and  \texttt{$channel$} is the number of the channel
ranging from
\texttt{0} to \texttt{31}.
\par\noindent Examples: \fbox{\texttt{H:0.0.0}}, \dots
\fbox{\texttt{H:1.11.31}}

\vskip 10pt
A HV channel is automatically associated to a specific pin number of
the output connector of the CAEN HV board. 

\vskip 10pt A external HV  harness  connecting a given HV board (Radial
connector) to the patch panel (Redel  Male S connector) uses an unique
ID  ranging from  \texttt{0} to  \texttt{23}.  We give to each
external HV harness a label with the following scheme:
\begin{center}
   \fbox{\texttt{E:$harness$}}
 \end{center}
where \texttt{$harness$} is the number of the external harness.
\par\noindent Examples: \fbox{\texttt{E:0}}, \dots
\fbox{\texttt{E:23}}

\vskip 10pt Compared with the  labelling scheme proposed by Mathieu in
original  documents,  it   has  been  decided  not   to  introduce  an
intermediate cable  identifier depending on  the location of  the PMTs
(main walls,  top row in  main walls,  X-walls, gamma veto rows).  A unique
scheme   is  used in place,   based   on  an   already  existing   informations,
independently of  the geometry.  Individual cables  within an external
harness are identified through the  pin numbers they are associated to
on  the   HV  board  output  connector.    This  pin/cable  identifier
propagates up  to the  patch panel  and beyond  to the  internal cable
terminations.  There is no need  to label internal cables because they
are   confined  within   their  harness   and  thus   never  addressed
individually during cabling operations.


\subsection{Optical modules}


The  identification scheme  of the  optical  modules and their PMTs
is  based on  the
addressing scheme defined in the geometry model and implemented in the
simulation     and     data    analysis     software\footnote{Falaise:
  \url{https://gitub.com/SuperNEMO-DBD/Falaise}}.     There   are    4
categories  of  optical  modules  and  thus  of  scintillator  blocks,
depending on their location in the experimental setup:

\begin{itemize}
  
\item Main wall block (Falaise: geometry category \texttt{"calorimeter\_block"}
  and  type  \texttt{1302}):

  \par  OMs are  addressed  through  their
  \emph{side}   number from 0 (Italy)     to  1 (France),
  \emph{column} number from 0 (Edelweiss) to 19 (Tunnel) and
  \emph{row}    number from 0 (bottom)    to 12 (top).

  \par We  propose to label such a block with the following scheme:
  \begin{center}
    \fbox{\texttt{M:$side$.$column$.$row$}}
  \end{center}
  \vskip 10pt
  \par\noindent Examples: \fbox{\texttt{M:0.0.0}}, 
  \fbox{\texttt{M:0.19.12}}, \fbox{\texttt{M:1.0.0}}, \fbox{\texttt{M:1.19.12}}.
  
\item X-wall block (Falaise: geometry category \texttt{"xcalo\_block"} and type
  \texttt{1232}):
  
  \par OMs  are addressed  through their
  \emph{side}   number from 0 (Italy)     to  1 (France),
  \emph{wall}   number from 0 (Edelweiss) to  1 (tunnel),
  \emph{column} number from 0 (source)    to  1 (calorimeter) and
  \emph{row}    number from 0 (bottom)    to 15 (top).
  
  \par  We propose  to
  label such a block with the following scheme:
  \begin{center}
    \fbox{\texttt{X:$side$.$wall$.$column$.$row$}}
  \end{center}
  \vskip 10pt
  \par\noindent Examples: \fbox{\texttt{X:0.1.1.15}}, \fbox{\texttt{X:1.0.0.8}}
  
\item Gamma veto block  (Falaise: geometry category \texttt{"gveto\_block"} and
  type   \texttt{1252}):
  
  \par   OMs  are   addressed  through   their
  \emph{side}   number from 0 (Italy)     to  1 (France),
  \emph{wall}   number from 0 (bottom)    to  1 (top)  and
  \emph{column} number from 0 (Edelweiss) to 15 (tunnel).
  
  \par We propose to label  such a block
  with the following scheme:
  \begin{center}
    \fbox{\texttt{G:$side$.$wall$.$column$}}
  \end{center}
  \vskip 10pt
  
  \par\noindent Examples: \fbox{\texttt{G:0.1.0}}, \fbox{\texttt{G:1.0.8}}
  
\item  Block for  reference  optical module:
  
  \par  OMs are  addressed  through their \emph{ref}  number.
  
  \par We propose to  label such a  block with the following scheme:
  \begin{center}
    \fbox{\texttt{R:$ref$}}
  \end{center}

\end{itemize}



\clearpage
\section{Patch panel routing}

The layout of the patch panel is extracted from DocDB-4663.
Figures \ref{fig:calohv:pp:0} and \ref{fig:calohv:pp:1}
show the routing of external and internal HV harnesses
on the patch panels for Italy and France sides.

\begin{figure}[h!]
  \begin{center}
    \scalebox{0.5}{\input{\pdftextimgpath/fig-calohv-pp-0.pdftex_t}}
  \end{center}
  \caption{Routing of HV harnesses on the patch panel (Italy-mountain side). Left: external view. Right: internal view.}
  \label{fig:calohv:pp:0}
\end{figure}

\begin{figure}[h!]
  \begin{center}
    \scalebox{0.5}{\input{\pdftextimgpath/fig-calohv-pp-1.pdftex_t}}
  \end{center}
  \caption{Routing of HV harnesses on the patch panel (France-mountain side). Left: internal view. Right: external view.}
  \label{fig:calohv:pp:1}
\end{figure}

%% end


\clearpage
\section{Rack}

The rack number 2 hosts the two CAEN HV Power Supply crates.
Figure \ref{fig:calohv:rack2:0} shows the position of both HV crates in rack 2.

\begin{figure}[h!]
  \begin{center}
    \scalebox{1}{\input{\pdftextimgpath/fig-calohv-rack2.pdftex_t}}
  \end{center}
  \caption{Position of HV crates in rack 2.}
  \label{fig:calohv:rack2:0}
\end{figure}


\clearpage
\section{The CaloHV crates}

Figures \ref{fig:calohv:crates:1}  and \ref{fig:calohv:crates:2}  show
respectively the repartition of boards within the CAEN HV Power Supply
crates 0 (Italy) and 1 (France).

\begin{figure}[h!]
  \begin{center}
    \scalebox{0.6}{\input{\pdftextimgpath/fig-calohv-2-italy.pdftex_t}}
  \end{center}
  \caption{CaloHV crate 0 (Italy).}
  \label{fig:calohv:crates:1}
\end{figure}

\begin{figure}[h!]
  \begin{center}
    \scalebox{0.6}{\input{\pdftextimgpath/fig-calohv-2-france.pdftex_t}}
  \end{center}
  \caption{CaloHV crate 1 (France).}
  \label{fig:calohv:crates:2}
 
\end{figure}

\clearpage

%% end

% -*- mode: latex; -*-

\clearpage
\section{The CaloHV cabling table and its usage}

\subsection{Source table}

Cabling the CaloHV system consists in the association of each PMT to a
CAEN HV  board channel.  An unique  cabling table is  provided to
give an unambiguous description of the  cable paths from the HV boards
to the  PMTs.  The table consists  in an associative map  like the one
shown on  table \ref{tab:calohv:map:1}.  This map contains  the needed
informations to ensure the addressing of all HV cables.

\begin{table}[h]
\begin{center}
\begin{tabular}{|c|c|c|c|}
  \hline
  \textbf{HV channel}& \textbf{External HV harness} & \textbf{Internal HV cable} & \textbf{Optical Module} \\
  \hline
  \hline
  \texttt{H:0.0.23}  & \texttt{E:12} & \texttt{A:12.3} & \texttt{M:0.19.0} \\
  \hline
  \texttt{H:0.0.22}  & \texttt{E:12} & \texttt{A:12.18} & \texttt{M:0.19.1} \\
  \hline
  \texttt{H:0.0.6}  & \texttt{E:12} & \texttt{A:12.34} & \texttt{M:0.19.2} \\
  \hline
  \texttt{H:0.2.23}  & \texttt{E:14} & \texttt{A:14.3} & \texttt{M:0.19.3} \\
  \hline
  \vdots          & \vdots  & \vdots   & \vdots             \\  
  \hline
\end{tabular}
\end{center}
\caption{Example of CaloHV cabling table}
\label{tab:calohv:map:1}
\end{table}

\par\noindent The table is provided in the form of a CSV\footnote{CSV:
  coma separated value} file.  The file must use the following format:

\begin{itemize}
\item The file contains only ASCII characters.
\item Blank lines are ignored.
\item Lines starting with the hashtag character \fbox{\texttt{\#}} are
  ignored, enabling to write some comments.
\item There is only one HV channel/PMT association per line.
\item Each  line has  four columns  separated by  the \emph{semi-colon}
  character \fbox{\texttt{;}}.
\item The first column contains the label of the CAEN HV board channel.
\item The second column contains the label of the external HV harness.
\item The third column contains the label of the internal HV cable.
\item The fourth column contains the label of the PMT (optical module).
\end{itemize}

\par\noindent The  CaloHV cabling map file  can be used as  the unique
source of information for different purposes:
\begin{itemize}
\item generation of labels to be stuck on internal HV cables, internal
  and external HV harnesses,
\item generation of  printable cabling tables for people  in charge of
  the calorimeter HV cabling at LSM,
\item input for dedicated software  modeling tools used by the Control
  and Monitoring System (CMS), the simulation\dots
\end{itemize}


\subsection{CaloHV cabling sheets}


The  SNCabling  package  provides  a  Python  script  to  automatically
generate, from the CaloHV cabling table, a printable PDF document with
cabling  tables  corresponding to  each  part  of the  detector and crates.
This document can be used during cabling operations.


\subsection{Labels}

The  SNCabling  package  provides  a Python  script  to  automatically
generate, from the  CaloHV cabling table, the lists of  all labels for
all  HV  harnesses and  cables.   The  labels  must  be stuck  on  the
terminations of all  harnesses and cables to help the  cabling team to
identify  the proper  connections between  CAEN HV  boards/external HV
harnesses/patch     panel/internal      HV     cables/PMT.      Figure
\ref{fig:calohv:labels:1}  shows where  various  kinds  of labels  are
supposed to be stuck on HV harnesses and cables.

\begin{itemize}
\item Each label  stuck on the end of an  internal HV cable on
the PMT  side identifies  not only  the HV cable  itself but  also the
optical module/PMT it is connected to. Example:
  \begin{center}
    \fbox{\texttt{A:9:3 -> M:1.3.4}}
  \end{center}
\item Each label  stuck on the end of an  external HV harness on
the HV crate  side identifies  not only  the HV harness  itself but  also the
HV board it is connected to. Example:
  \begin{center}
    \fbox{\texttt{E:9 -> B:1.9}}
  \end{center}
\end{itemize}

\begin{figure}[h!]
  \begin{center}
    \scalebox{0.75}{\input{\pdftextimgpath/fig-calohv-labels-1.pdftex_t}}
  \end{center}
  \caption{CaloHV labelling of HV harnesses and internal cables}
  \label{fig:calohv:labels:1}
\end{figure}

%% end


\end{document}
