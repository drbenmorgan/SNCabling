\documentclass[12pt,a4paper]{article}

\usepackage{a4wide}
\usepackage{verbatim}
\usepackage[T1]{fontenc}
\usepackage{ucs}
\usepackage[utf8x]{inputenc}
%\usepackage[french]{babel}
\usepackage[dvips]{color}
\usepackage{graphicx}
\usepackage{epic}
\usepackage{eepic}
\usepackage{eepicemu}
\usepackage{array}
\usepackage{moreverb}
\usepackage{fancyvrb}
\usepackage{url}
\usepackage{eurosym}
\usepackage{amsmath}
\usepackage{amssymb}
\usepackage{multicol}

\newcommand{\imgpath}{./images}
\newcommand{\samplespath}{./samples}
\newcommand{\pdfteximgpath}{./pdftex}
\newcommand{\pdftextimgpath}{./pdftex_t}

\title{SuperNEMO Demonstrator\\
  Light Injection System (LIS)\\
  Cabling scheme and cable labels\\
  version 0.2}
\author{Y.Lemi\`ere, F.Mauger}
\date{October 5th, 2018}

%%%%%%%%%%%%%%%%
\begin{document}

\maketitle

\begin{abstract}
  \noindent This document presents the cable labelling convention used
  for the SuperNEMO Demonstrator's Light Injection System (LIS).
\end{abstract}

\tableofcontents

\section{Principle}

The  Light  Injection  System  implements  20  LEDs  which  distribute
reference  pulses  of  light  to  all  optical  modules  (OM)  in  the
demonstrator (main calorimeter walls, X-walls,  gamma veto) as well as
to a few reference OMs.

Each LED is  able to send light  through an unique bundle  of about 70
optical fibers.  The termination of each  fiber must be connected to a
scintillator block from OM in  the main calorimeter walls, the X-walls
or the gamma veto setup.

Each  scintillator block  has  two  connectors in  order  to plug  the
optical fibers which  transport light from LEDs to  the detector.  One
of  this connector  is named  the \emph{primary}  connector while  the
other is the \emph{secondary} connector.  The \emph{primary} connector
is intended to receive light pulse  from a LED for routine calibration
operations. The \emph{secondary} connector is  used as a spare in case
of failure of the \emph{primary} line.

\vskip 10pt
\noindent\textbf{Note:}  We  have  no   information  today  about  the
identification  of   the  OM's  \emph{primary}   and  \emph{secondary}
connectors  with respect  to their  placement and  orientation in  the
detector.
\vskip 10pt

\noindent
Figure \ref{fig:lis:principle:1}  shows the  basic diagram of  the LIS
system.  For  practical reason  one considers to  use a  special label
stuck on each optical fiber and bundle during cabling operations.

\begin{figure}[h!]
  \begin{center}
    \scalebox{0.75}{\input{\pdftextimgpath/fig-lis-1.pdftex_t}}
  \end{center}
  \caption{Principle of  the light distribution  to the OMs  using the
    LIS and labelling.}
  \label{fig:lis:principle:1}
\end{figure}

\pagebreak
\section{Addressing objects}

\subsection{Format of a LIS label}

Each label to be used for LIS cabling will use the following format:

\begin{center}
  \fbox{\texttt{X:}$id_1$.$id_2$\dots$id_n$}
\end{center}
\noindent  where \texttt{X}  is a  single letter  which describes  the
category  of the  labelled  object,  and the  $id_1$.$id_2$\dots$id_n$
sequence is the unique address of the object within its category.  The
$id_x$ tokens  are positive  integers.  The \emph{colon}  character is
used to separate  the category letter from the  address.  The sequence
of  identifiers in  the  address  use the  \emph{dot}  character as  a
separator.


\subsection{LEDs, bundles and fibers}

Each LED belonging  to the LI system is identified  with an unique ID,
namely a number ranging from \texttt{1} to  \texttt{20}.
We propose to label a given LED
with the following scheme:
\begin{center}
  \fbox{\texttt{L:$led$}}
\end{center}
where \texttt{$led$} is the number of the LED (positive integer).
\vskip 10pt
\par\noindent Examples: \fbox{\texttt{L:3}}, 
\fbox{\texttt{L:4}},\fbox{\texttt{L:12}}.
\vskip 10pt
\par Practically, the LED labels should not be necessary unless
they can easily be stuck on the LIS crate.

\vskip 10pt

\par\noindent Up  to 70 optical fibers  are associated to a  LED. They
all are grouped within a single  bundle.  A bundle inherits the number
of the  LED it is  connected to.  We propose  to label a  given bundle
with the following scheme:
\begin{center}
  \fbox{\texttt{B:$led$}}
\end{center}
where \texttt{$led$} is the number of the LED associated to the bundle.
\vskip 10pt
\par\noindent Examples: \fbox{\texttt{B:3}}, 
\fbox{\texttt{B:4}}\dots\fbox{\texttt{B:12}}.
 
\vskip 10pt  An optical  fiber inherits  the number of  the LED  it is
associated to.  It is distinguished from  the other fibers in the same
bundle thanks  to an  additional \emph{fiber  number}.  We  propose to
label a  given fiber connected  to the block's primary  connector with
the following scheme:
\begin{center}
  \fbox{\texttt{P:$led$.$fib$}}\hskip 1cm or\hskip 1cm\fbox{\texttt{S:$led$.$fib$}}
\end{center}
where \texttt{$led$} is the number  of the LED (and bundle) associated
to the fiber and \texttt{$fib$} is  the number of the fiber within the
bundle  (starting at  1). The  leading \texttt{P}  is used  for fibers
connected to the \emph{primary} connector of a scintillator block; the
\texttt{S}  is  used  for  fibers connected  to  the  \emph{secondary}
connector of a scintillator block.
\vskip 10pt
\par\noindent Examples: \fbox{\texttt{P:3.1}}, 
\fbox{\texttt{P:3.34}}, \fbox{\texttt{S:12.3}}.

\subsection{Optical modules}

The  identification scheme  of the  optical  modules is  based on  the
addressing scheme defined in the geometry model and implemented in the
simulation     and     data    analysis     software\footnote{Falaise:
  \url{https://gitub.com/SuperNEMO-DBD/Falaise}}.     There   are    4
categories  of  optical  modules  and  thus  of  scintillator  blocks,
depending on their location in the experimental setup:

\begin{itemize}
  
\item Main wall block (Falaise: geometry category \texttt{"calorimeter\_block"}
  and  type  \texttt{1302}):

  \par  OMs are  addressed  through  their
  \emph{side}   number from 0 (Italy)     to  1 (France),
  \emph{column} number from 0 (Edelweiss) to 19 (Tunnel) and
  \emph{row}    number from 0 (bottom)    to 12 (top).

  \par We  propose to label such a block with the following scheme:
  \begin{center}
    \fbox{\texttt{M:$side$.$column$.$row$}}
  \end{center}
  \vskip 10pt
  \par\noindent Examples: \fbox{\texttt{M:0.0.0}}, 
  \fbox{\texttt{M:0.19.12}}, \fbox{\texttt{M:1.0.0}}, \fbox{\texttt{M:1.19.12}}.
  
\item X-wall block (Falaise: geometry category \texttt{"xcalo\_block"} and type
  \texttt{1232}):
  
  \par OMs  are addressed  through their
  \emph{side}   number from 0 (Italy)     to  1 (France),
  \emph{wall}   number from 0 (Edelweiss) to  1 (tunnel),
  \emph{column} number from 0 (source)    to  1 (calorimeter) and
  \emph{row}    number from 0 (bottom)    to 15 (top).
  
  \par  We propose  to
  label such a block with the following scheme:
  \begin{center}
    \fbox{\texttt{X:$side$.$wall$.$column$.$row$}}
  \end{center}
  \vskip 10pt
  \par\noindent Examples: \fbox{\texttt{X:0.1.1.15}}, \fbox{\texttt{X:1.0.0.8}}
  
\item Gamma veto block  (Falaise: geometry category \texttt{"gveto\_block"} and
  type   \texttt{1252}):
  
  \par   OMs  are   addressed  through   their
  \emph{side}   number from 0 (Italy)     to  1 (France),
  \emph{wall}   number from 0 (bottom)    to  1 (top)  and
  \emph{column} number from 0 (Edelweiss) to 15 (tunnel).
  
  \par We propose to label  such a block
  with the following scheme:
  \begin{center}
    \fbox{\texttt{G:$side$.$wall$.$column$}}
  \end{center}
  \vskip 10pt
  
  \par\noindent Examples: \fbox{\texttt{G:0.1.0}}, \fbox{\texttt{G:1.0.8}}
  
\item  Block for  reference  optical module:
  
  \par  OMs are  addressed  through their \emph{ref}  number.
  
  \par We propose to  label such a  block with the following scheme:
  \begin{center}
    \fbox{\texttt{R:$ref$}}
  \end{center}

\end{itemize}


\section{The LIS cabling table and its usage}

\subsection{Source table}

Cabling the LIS consists in  the association of each OM's scintillator
block to one  or two optical fibers.  An unique  cabling table must be
provided to give  an unambiguous description of the  fibers' path from
the LIS boards to the detector.   The table consists in an associative
map like the one shown on table \ref{tab:lis:map:1}.

\begin{table}[h]
\begin{center}
\begin{tabular}{|c|c|}
  \hline
  \textbf{Fiber label} & \textbf{Scintillator block label} \\
  \hline
  \hline
  \multicolumn{2}{|l|}{\textbf{Bundle} \texttt{B:1} (primary)}  \\
  \hline
  \texttt{P:1.1}  & \texttt{M:0.1.0} \\
  \hline
  \texttt{P:1.2}  & \texttt{M:0.2.0} \\
  \hline
  \texttt{P:1.3}  & \texttt{M:0.3.0} \\
  \hline
  \vdots          & \vdots             \\  
  \hline
  \hline
  \multicolumn{2}{|l|}{\textbf{Bundle} \texttt{B:2} (primary)}  \\
  \hline
  \texttt{P:2.1}  & \texttt{M:0.6.0} \\
  \hline
  \texttt{P:2.2}  & \texttt{M:0.7.0} \\
  \hline
  \vdots          & \vdots             \\  
  \hline
  \hline
  \multicolumn{2}{|l|}{\textbf{Bundle} \texttt{B:12} (secondary)}  \\
  \hline
  \vdots          & \vdots             \\  
  \hline
  \texttt{S:12.5}  & \texttt{X:1.15.8.2} \\
  \hline
  \vdots          & \vdots             \\  
  \hline
\end{tabular}
\end{center}
\caption{Example of LIS cabling table}
\label{tab:lis:map:1}
\end{table}

\par\noindent The  Austin group, in  charge of  the LI system,  has to
provide this table  in the form of a  CSV\footnote{CSV: coma separated
  value} file. The file must use the following format:

\begin{itemize}
\item The file contains only ASCII characters.
\item Blank lines are ignored.
\item Lines starting with the hashtag character \fbox{\texttt{\#}} are
  ignored, enabling to write some comments.
\item There is only one fiber/OM association per line.
\item Each  line has  two columns  separated by  the \emph{semi-colon}
  character \fbox{\texttt{;}}.
\item The first column contains the label of an optical fiber (primary
  or secondary).
\item The second column contains the  label of an optical module (main
  wall, x-wall, gamma veto or reference OM).
\end{itemize}
\par\noindent  With  this  system,  a  given  optical  module  can  be
addressed  twice  because  of  its connections  to  both  primary  and
secondary LIS fibers.

\par\noindent The  LIS cabling  map file  will be  used as  the unique
source of information for different purposes:
\begin{itemize}
\item generation of labels to be stuck on fibers and bundles,
\item generation of  printable tables for people in charge  of the LIS
  cabling at LSM,
\item input for dedicated software  modeling tools used by the
  Control and Monitoring System (CMS), the simulation\dots 
\end{itemize}

\par\noindent An extract of the expected LIS cabling map file is shown
on table \ref{tab:lis:map:2}.
\begin{table}[h]
  \begin{center}
    \begin{tabular}{|ccc|}
      \hline
      \multicolumn{3}{|c|}{\vdots}\\
      \texttt{P:}$led_a$\texttt{.}$fib_a$  & \texttt{;} & \texttt{M:1.12.11} \\
      \texttt{S:}$led_b$\texttt{.}$fib_b$  & \texttt{;} & \texttt{M:1.12.11} \\
      \multicolumn{3}{|c|}{\vdots}\\
      \hline
    \end{tabular}
  \end{center}
  \caption{Extract  of  the  LIS  cabling map file.   Symbol
    $led_X$ corresponds  to the identifier  of a LED.   Symbol $fib_X$
    corresponds  to  the  identifier  of   an  optical  fiber  in  its
    bundle. In  this example,  the scintillator  block of  the optical
    module on  French side, column 12  and row 11 is  connected to two
    fibers (one  primary and  one secondary) lit  up by  two different
    LEDs.}
\label{tab:lis:map:2}
\end{table}


\subsection{LIS cabling sheets}

From  the LIS  cabling  table, a  Python script  will  be provided  to
generate a printable PDF document with cabling tables corresponding to
each  part of  the  detector.  A typical  output  is  shown on  figure
\ref{fig:lis:sheet:1}.


\begin{figure}[h!]
  \begin{center}
    \scalebox{0.58}{\begin{tabular}{|r||c|c|c|c|c|c|c|c|c|c|c|c|c|c|c|c|c|c|c|c||l|}
\hline
 & \textcolor{blue}{\small 19} & \textcolor{blue}{\small 18} & \textcolor{blue}{\small 17} & \textcolor{blue}{\small 16} & \textcolor{blue}{\small 15} & \textcolor{blue}{\small 14} & \textcolor{blue}{\small 13} & \textcolor{blue}{\small 12} & \textcolor{blue}{\small 11} & \textcolor{blue}{\small 10} & \textcolor{blue}{\small 9} & \textcolor{blue}{\small 8} & \textcolor{blue}{\small 7} & \textcolor{blue}{\small 6} & \textcolor{blue}{\small 5} & \textcolor{blue}{\small 4} & \textcolor{blue}{\small 3} & \textcolor{blue}{\small 2} & \textcolor{blue}{\small 1} & \textcolor{blue}{\small 0} & \\
\hline
\hline
\textcolor{blue}{\small 12} & P:?.?  &  P:?.?  &  P:?.?  &  P:?.?  &  P:?.?  &  P:?.?  &  P:?.?  &  P:?.?  &  P:?.?  &  P:?.?  &  P:?.?  &  P:?.?  &  P:?.?  &  P:?.?  &  P:?.?  &  P:?.?  &  P:?.?  &  P:?.?  &  P:?.?  &  P:?.? & \textcolor{blue}{\small 12}\\
 & S:?.?  &  S:?.?  &  S:?.?  &  S:?.?  &  S:?.?  &  S:?.?  &  S:?.?  &  S:?.?  &  S:?.?  &  S:?.?  &  S:?.?  &  S:?.?  &  S:?.?  &  S:?.?  &  S:?.?  &  S:?.?  &  S:?.?  &  S:?.?  &  S:?.?  &  S:?.? & \\
\hline
\textcolor{blue}{\small 11} & P:?.?  &  P:?.?  &  P:?.?  &  P:?.?  &  P:?.?  &  P:?.?  &  P:?.?  &  P:?.?  &  P:?.?  &  P:?.?  &  P:?.?  &  P:?.?  &  P:?.?  &  P:?.?  &  P:?.?  &  P:?.?  &  P:?.?  &  P:?.?  &  P:?.?  &  P:?.? & \textcolor{blue}{\small 11}\\
 & S:?.?  &  S:?.?  &  S:?.?  &  S:?.?  &  S:?.?  &  S:?.?  &  S:?.?  &  S:?.?  &  S:?.?  &  S:?.?  &  S:?.?  &  S:?.?  &  S:?.?  &  S:?.?  &  S:?.?  &  S:?.?  &  S:?.?  &  S:?.?  &  S:?.?  &  S:?.? & \\
\hline
\textcolor{blue}{\small 10} & P:?.?  &  P:?.?  &  P:?.?  &  P:?.?  &  P:?.?  &  P:?.?  &  P:?.?  &  P:?.?  &  P:?.?  &  P:?.?  &  P:?.?  &  P:?.?  &  P:?.?  &  P:?.?  &  P:?.?  &  P:?.?  &  P:?.?  &  P:?.?  &  P:?.?  &  P:?.? & \textcolor{blue}{\small 10}\\
 & S:?.?  &  S:?.?  &  S:?.?  &  S:?.?  &  S:?.?  &  S:?.?  &  S:?.?  &  S:?.?  &  S:?.?  &  S:?.?  &  S:?.?  &  S:?.?  &  S:?.?  &  S:?.?  &  S:?.?  &  S:?.?  &  S:?.?  &  S:?.?  &  S:?.?  &  S:?.? & \\
\hline
\textcolor{blue}{\small 9} & P:?.?  &  P:?.?  &  P:?.?  &  P:?.?  &  P:?.?  &  P:?.?  &  P:?.?  &  P:?.?  &  P:?.?  &  P:?.?  &  P:?.?  &  P:?.?  &  P:?.?  &  P:?.?  &  P:?.?  &  P:?.?  &  P:?.?  &  P:?.?  &  P:?.?  &  P:?.? & \textcolor{blue}{\small 9}\\
 & S:?.?  &  S:?.?  &  S:?.?  &  S:?.?  &  S:?.?  &  S:?.?  &  S:?.?  &  S:?.?  &  S:?.?  &  S:?.?  &  S:?.?  &  S:?.?  &  S:?.?  &  S:?.?  &  S:?.?  &  S:?.?  &  S:?.?  &  S:?.?  &  S:?.?  &  S:?.? & \\
\hline
\textcolor{blue}{\small 8} & P:?.?  &  P:?.?  &  P:?.?  &  P:?.?  &  P:?.?  &  P:?.?  &  P:?.?  &  P:?.?  &  P:?.?  &  P:?.?  &  P:?.?  &  P:?.?  &  P:?.?  &  P:?.?  &  P:?.?  &  P:?.?  &  P:?.?  &  P:?.?  &  P:?.?  &  P:?.? & \textcolor{blue}{\small 8}\\
 & S:?.?  &  S:?.?  &  S:?.?  &  S:?.?  &  S:?.?  &  S:?.?  &  S:?.?  &  S:?.?  &  S:?.?  &  S:?.?  &  S:?.?  &  S:?.?  &  S:?.?  &  S:?.?  &  S:?.?  &  S:?.?  &  S:?.?  &  S:?.?  &  S:?.?  &  S:?.? & \\
\hline
\textcolor{blue}{\small 7} & P:?.?  &  P:?.?  &  P:?.?  &  P:?.?  &  P:?.?  &  P:?.?  &  P:?.?  &  P:?.?  &  P:?.?  &  P:?.?  &  P:?.?  &  P:?.?  &  P:?.?  &  P:?.?  &  P:?.?  &  P:?.?  &  P:?.?  &  P:?.?  &  P:?.?  &  P:?.? & \textcolor{blue}{\small 7}\\
 & S:?.?  &  S:?.?  &  S:?.?  &  S:?.?  &  S:?.?  &  S:?.?  &  S:?.?  &  S:?.?  &  S:?.?  &  S:?.?  &  S:?.?  &  S:?.?  &  S:?.?  &  S:?.?  &  S:?.?  &  S:?.?  &  S:?.?  &  S:?.?  &  S:?.?  &  S:?.? & \\
\hline
\textcolor{blue}{\small 6} & P:?.?  &  P:?.?  &  P:?.?  &  P:?.?  &  P:?.?  &  P:?.?  &  P:?.?  &  P:?.?  &  P:?.?  &  P:?.?  &  P:?.?  &  P:?.?  &  P:?.?  &  P:?.?  &  P:?.?  &  P:?.?  &  P:?.?  &  P:?.?  &  P:?.?  &  P:?.? & \textcolor{blue}{\small 6}\\
 & S:?.?  &  S:?.?  &  S:?.?  &  S:?.?  &  S:?.?  &  S:?.?  &  S:?.?  &  S:?.?  &  S:?.?  &  S:?.?  &  S:?.?  &  S:?.?  &  S:?.?  &  S:?.?  &  S:?.?  &  S:?.?  &  S:?.?  &  S:?.?  &  S:?.?  &  S:?.? & \\
\hline
\textcolor{blue}{\small 5} & P:?.?  &  P:?.?  &  P:?.?  &  P:?.?  &  P:?.?  &  P:?.?  &  P:?.?  &  P:?.?  &  P:?.?  &  P:?.?  &  P:?.?  &  P:?.?  &  P:?.?  &  P:?.?  &  P:?.?  &  P:?.?  &  P:?.?  &  P:?.?  &  P:?.?  &  P:?.? & \textcolor{blue}{\small 5}\\
 & S:?.?  &  S:?.?  &  S:?.?  &  S:?.?  &  S:?.?  &  S:?.?  &  S:?.?  &  S:?.?  &  S:?.?  &  S:?.?  &  S:?.?  &  S:?.?  &  S:?.?  &  S:?.?  &  S:?.?  &  S:?.?  &  S:?.?  &  S:?.?  &  S:?.?  &  S:?.? & \\
\hline
\textcolor{blue}{\small 4} & P:?.?  &  P:?.?  &  P:?.?  &  P:?.?  &  P:?.?  &  P:?.?  &  P:?.?  &  P:?.?  &  P:?.?  &  P:?.?  &  P:?.?  &  P:?.?  &  P:?.?  &  P:?.?  &  P:?.?  &  P:?.?  &  P:?.?  &  P:?.?  &  P:?.?  &  P:?.? & \textcolor{blue}{\small 4}\\
 & S:?.?  &  S:?.?  &  S:?.?  &  S:?.?  &  S:?.?  &  S:?.?  &  S:?.?  &  S:?.?  &  S:?.?  &  S:?.?  &  S:?.?  &  S:?.?  &  S:?.?  &  S:?.?  &  S:?.?  &  S:?.?  &  S:?.?  &  S:?.?  &  S:?.?  &  S:?.? & \\
\hline
\textcolor{blue}{\small 3} & P:?.?  &  P:?.?  &  P:?.?  &  P:?.?  &  P:?.?  &  P:?.?  &  P:?.?  &  P:?.?  &  P:?.?  &  P:?.?  &  P:?.?  &  P:?.?  &  P:?.?  &  P:?.?  &  P:?.?  &  P:?.?  &  P:?.?  &  P:?.?  &  P:?.?  &  P:?.? & \textcolor{blue}{\small 3}\\
 & S:?.?  &  S:?.?  &  S:?.?  &  S:?.?  &  S:?.?  &  S:?.?  &  S:?.?  &  S:?.?  &  S:?.?  &  S:?.?  &  S:?.?  &  S:?.?  &  S:?.?  &  S:?.?  &  S:?.?  &  S:?.?  &  S:?.?  &  S:?.?  &  S:?.?  &  S:?.? & \\
\hline
\textcolor{blue}{\small 2} & P:?.?  &  P:?.?  &  P:?.?  &  P:?.?  &  P:?.?  &  P:?.?  &  P:?.?  &  P:?.?  &  P:?.?  &  P:?.?  &  P:?.?  &  P:?.?  &  P:?.?  &  P:?.?  &  P:?.?  &  P:?.?  &  P:?.?  &  P:?.?  &  P:?.?  &  P:?.? & \textcolor{blue}{\small 2}\\
 & S:?.?  &  S:?.?  &  S:?.?  &  S:?.?  &  S:?.?  &  S:?.?  &  S:?.?  &  S:?.?  &  S:?.?  &  S:?.?  &  S:?.?  &  S:?.?  &  S:?.?  &  S:?.?  &  S:?.?  &  S:?.?  &  S:?.?  &  S:?.?  &  S:?.?  &  S:?.? & \\
\hline
\textcolor{blue}{\small 1} & P:?.?  &  P:?.?  &  P:?.?  &  P:?.?  &  P:?.?  &  P:?.?  &  P:?.?  &  P:?.?  &  P:?.?  &  P:?.?  &  P:?.?  &  P:?.?  &  P:?.?  &  P:?.?  &  P:?.?  &  P:?.?  &  P:?.?  &  P:?.?  &  P:?.?  &  P:?.? & \textcolor{blue}{\small 1}\\
 & S:?.?  &  S:?.?  &  S:?.?  &  S:?.?  &  S:?.?  &  S:?.?  &  S:?.?  &  S:?.?  &  S:?.?  &  S:?.?  &  S:?.?  &  S:?.?  &  S:?.?  &  S:?.?  &  S:?.?  &  S:?.?  &  S:?.?  &  S:?.?  &  S:?.?  &  S:?.? & \\
\hline
\textcolor{blue}{\small 0} & P:?.?  &  P:?.?  &  P:?.?  &  P:?.?  &  P:?.?  &  P:?.?  &  P:?.?  &  P:?.?  &  P:?.?  &  P:?.?  &  P:?.?  &  P:?.?  &  P:?.?  &  P:?.?  &  P:?.?  &  P:?.?  &  P:?.?  &  P:?.?  &  P:?.?  &  P:?.? & \textcolor{blue}{\small 0}\\
 & S:?.?  &  S:?.?  &  S:?.?  &  S:?.?  &  S:?.?  &  S:?.?  &  S:?.?  &  S:?.?  &  S:?.?  &  S:?.?  &  S:?.?  &  S:?.?  &  S:?.?  &  S:?.?  &  S:?.?  &  S:?.?  &  S:?.?  &  S:?.?  &  S:?.?  &  S:?.? & \\
\hline
\hline
 & \textcolor{blue}{\small 19} & \textcolor{blue}{\small 18} & \textcolor{blue}{\small 17} & \textcolor{blue}{\small 16} & \textcolor{blue}{\small 15} & \textcolor{blue}{\small 14} & \textcolor{blue}{\small 13} & \textcolor{blue}{\small 12} & \textcolor{blue}{\small 11} & \textcolor{blue}{\small 10} & \textcolor{blue}{\small 9} & \textcolor{blue}{\small 8} & \textcolor{blue}{\small 7} & \textcolor{blue}{\small 6} & \textcolor{blue}{\small 5} & \textcolor{blue}{\small 4} & \textcolor{blue}{\small 3} & \textcolor{blue}{\small 2} & \textcolor{blue}{\small 1} & \textcolor{blue}{\small 0} & \\
\hline
\end{tabular}
}
  \end{center}
  \caption{Expected  printable   LIS  cabling  table  for   the  Italy
    calorimeter  main  wall  (front  view).  Question  marks  will  be
    replaced by the real fibers' identifiers when the final table will
    be  available. Similar  tables will  be provided  for X-walls  and
    gamma veto lines.}
  \label{fig:lis:sheet:1}
\end{figure}


\subsection{LIS labels}

Practically, we  plan to generate  labels for all bundles  and optical
fibers  to  help  the  cabling  team  to  identifiy  the  proper  OM's
connection for each fiber.  The label  will be prepared and stuck near
the termination of the fiber. We propose that the label shows both the
fiber identifier and the OM's identifier.

\noindent Example below displays the  full text expected to be printed
on a fiber label:
\begin{center}
\fbox{\texttt{P:2.34} $\rightarrow$ \texttt{M:0.2.5}}
\end{center}

\par\noindent Such  informations are of course  redundant with respect
to  the  printable   plain  table  shown  in   the  previous  section.
The  fiber labels could  possibly be removed before  installing the
coil.

\par\noindent We  consider to write  a Python script  to automatically
generate all labels (up to 1400).

\end{document}
