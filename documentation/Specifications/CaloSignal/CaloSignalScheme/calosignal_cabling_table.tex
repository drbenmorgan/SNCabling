% -*- mode: latex; -*-

\clearpage
\section{The Calorimeter signal cabling table and its usage}

\subsection{Source table}

Cabling the calorimeter readout system  consists in the association of
each  PMT  to a  Wavecatcher  channel.   An  unique cabling  table  is
provided to give an unambiguous  description of the signal cable paths
from  the calorimeter  front-end boards  to the  detector.  The  table
consists  in  an   associative  map  like  the  one   shown  on  table
\ref{tab:calosignal:map:1}.

\begin{table}[h]
\begin{center}
\begin{tabular}{|c|c|c|c|}
  \hline
  \textbf{Readout Channel}& \textbf{External signal cable} & \textbf{Internal signal cable} & \textbf{Optical Module} \\
  \hline
  \hline
  \texttt{H:0.0.0}   & \texttt{L:11.0}   & \texttt{A:11.0}   & \texttt{M:0.0.0}   \\
  \hline
  \texttt{H:0.0.1}   & \texttt{L:11.20}  & \texttt{A:11.20}  & \texttt{M:0.0.1}   \\
  \hline
  \texttt{H:0.0.2}   & \texttt{L:12.0}   & \texttt{A:12.0}   & \texttt{M:0.0.2}   \\
  \hline
  \vdots             & \vdots            &  \vdots           & \vdots             \\  
  \hline
\end{tabular}
\end{center}
\caption{Example of CaloSignal cabling table}
\label{tab:calosignal:map:1}
\end{table}

\par\noindent The table is provided in the form of a CSV\footnote{CSV:
  coma separated value} file.  The file uses the following format:

\begin{itemize}
\item The file contains only ASCII characters.
\item Blank lines are ignored.
\item Lines starting with the hashtag character \fbox{\texttt{\#}} are
  ignored, enabling to write some comments.
\item  There  is  only  one Wavecatcher  front-end  board  channel/PMT
  association per line.
\item Each  line has four  columns separated by  the \emph{semi-colon}
  character \fbox{\texttt{;}}.
\item The first column contains the label of the Wavecatcher front-end
  board readout channel.
\item The second column contains the label of the external signal cable.
\item The third column contains the label of the internal signal cable.
\item  The  fourth column  contains  the  label  of the  PMT  (optical
  module).
\end{itemize}

\par\noindent The  CaloSignal cabling map  file is used as  the unique
source of information for different purposes:
\begin{itemize}
\item generation of labels to be stuck on external and internal cables
  or harnesses;
\item  generation of  printable tables  for  people in  charge of  the
  calorimeter readout cabling at LSM,
\item input for dedicated software  modeling tools used by the Control
  and Monitoring System (CMS), the simulation\dots
\end{itemize}


\subsection{CaloSignal cabling sheets}


The  SNCabling  package  provides  a Python  script  to  automatically
generate, from the CaloSignal cabling  table, a printable PDF document
with cabling  tables corresponding  to each part  of the  detector and
each front-end crate.

\subsection{Labels}

The  SNCabling  package  provides  a Python  script  to  automatically
generate, from the  CaloSignal cabling table, the lists of  all labels for all
signal  harnesses  and  cables.   The  labels must  be  stuck  on  the
terminations of all  harnesses and cables to help the  cabling team to
identifiy   the    proper   connections   between    front-end   board
channels/external  cables/patch   panel/internal  cables/PMT.   Figure
\ref{fig:calosignal:principle:1} shows  where various kinds  of labels
are supposed to be stuck on signal harnesses and cables.

\begin{itemize}
\item Each label  stuck on the end of an  internal signal cable on
  the PMT  side identifies  not only  the signal cable  itself but  also the
  optical module/PMT it is connected to. Example:
  \begin{center}
    \fbox{\texttt{A:18.15 -> X:0.0.0.15}}
  \end{center}
\item Each label  stuck on the end of an  external signal cable on
  the front-end crate  side identifies  not only  the signal cable  itself but  also the
  Wavecatcher readout channel it is connected to.Example:
  \begin{center}
    \fbox{\texttt{L:18.15 -> H:2.6.15}}
  \end{center}
\end{itemize}

%% end
