% -*- mode: latex; -*-

\section{The Calorimeter signal cabling table and its usage}

\subsection{Source table}

Cabling the calorimeter readout system  consists in the association of
each PMT  to a Wavecatcher channel.   An unique cabling table  must be
provided to  give an unambiguous  description of the cable  paths from
the calorimeter front-end boards to  the detector.  The table consists
in   an    associative   map   like    the   one   shown    on   table
\ref{tab:calosignal:map:1}.

\begin{table}[h]
\begin{center}
\begin{tabular}{|c|c|c|c|}
  \hline
  \textbf{Channel}& \textbf{External cable} & \textbf{Internal cable} & \textbf{Optical Module} \\
  \hline
  \hline
  \texttt{H:0.0.0}   & \texttt{L:11.0}   & \texttt{A:11.0}   & \texttt{M:0.0.0}   \\
  \hline
  \texttt{H:0.0.1}   & \texttt{L:11.20}  & \texttt{A:11.20}  & \texttt{M:0.0.1}   \\
  \hline
  \texttt{H:0.0.2}   & \texttt{L:12.0}   & \texttt{A:12.0}   & \texttt{M:0.0.2}   \\
  \hline
  \vdots             & \vdots            &  \vdots           & \vdots             \\  
  \hline
\end{tabular}
\end{center}
\caption{Example of CaloSignal cabling table}
\label{tab:calosignal:map:1}
\end{table}

\par\noindent The table is provided in the form of a CSV\footnote{CSV:
  coma separated value} file.  The file uses the following format:

\begin{itemize}
\item The file contains only ASCII characters.
\item Blank lines are ignored.
\item Lines starting with the hashtag character \fbox{\texttt{\#}} are
  ignored, enabling to write some comments.
\item  There  is  only  one Wavecatcher  front-end  board  channel/PMT
  association per line.
\item Each  line has four  columns separated by  the \emph{semi-colon}
  character \fbox{\texttt{;}}.
\item The first column contains the label of the Wavecatcher front-end
  board channel.
\item The second column contains the label of the external cable.
\item The third column contains the label of the internal cable.
\item  The  fourth column  contains  the  label  of the  PMT  (optical
  module).
\end{itemize}

\par\noindent  The CaloSignal  cabling map  file  can be  used as  the
unique source of information for different purposes:
\begin{itemize}
\item generation of labels to be stuck on external and internal cables
  or harnesses;
\item  generation of  printable tables  for  people in  charge of  the
  calorimeter readout cabling at LSM,
\item input for dedicated software  modeling tools used by the Control
  and Monitoring System (CMS), the simulation\dots
\end{itemize}


\subsection{CaloSignal cabling sheets}


A  Python  script is  provided  to  automatically generate,  from  the
CaloSignal cabling table, a printable PDF document with cabling tables
corresponding to each part of the detector.

\subsection{Labels}

A  Python  script is  provided  to  automatically generate,  from  the
CaloSignal cabling table, lists of labels for all signal harnesses and
cables.  The labels must be stuck on the terminations of all harnesses
and  cables  to  help  the   cabling  team  to  identifiy  the  proper
connections  between  front-end board  channels/external  cables/patch
panel/internal  cables/PMT.   Figure  \ref{fig:calosignal:principle:1}
shows  where various  kinds  of labels  are supposed  to  be stuck  on
harnesses and cables.

%% end
